\chapter{sPHENIX Commissioning}
\label{chap:commissioning}

As expected with any new collider detector, there will be significant commissioning time required in order to be ready for physics quality data.   The sPHENIX project and collaboration teams have taken this task seriously with detailed input from the detector subsystems and the sPHENIX Physics Working Groups.    We detail the initial commissioning plan and timeline for 2023 in this Chapter.

\section{Timeline}

\begin{table}[]
    \centering
    \begin{tabular}{|c|c|} \hline
        Weeks & Details \\ \hline \hline
        1.5 & Low rate, 6 bunches \\ \hline
        2.0 & Low rate, 110 bunches, MBD L1 timing \\ \hline
        1.0 & Low rate, crossing angle checks \\ \hline
        1.0 & Low rate, calorimeter timing \\ \hline
        4.0 & medium rate, TPC timing, optimization \\ \hline
        2.0 & full rate, system test, DAQ throughput \\ \hline \hline
        11.5 total & \\ \hline
    \end{tabular}
    \caption{Timeline for sPHENIX commissioning period in 2023, the first year of operation.}
    \label{tab:commision}
\end{table}

Commissioning sPHENIX with beams in RHIC should progress in stages with gradually
increasing luminosity.  \auau collisions are necessary for  commissioning due
to the high multiplicities which can be used to assess the detector performance with 
larger occupancy.   We highlight that installation of the sPHENIX inner silicon detectors takes 2-3 weeks and thus these sensitive detectors will be installed ahead of the 2023 running period for system debugging and cosmic ray data taking.   The presence of the full detector configuration means that very careful monitoring of luminosity and beam conditions is essential to maintain stable operations and minimize detector risk.   Thus, the sPHENIX experiment will require very careful coordination with C-AD including potential accelerator down times for access.    A summary of the projected initial commissioning timeline is shown in Table~\ref{tab:commision}.

\section{Commission Period Details}

For the purposes of this Beam Use Proposal, we include a very brief summary of the timeline as given.    
Except for initial stores with six bunches, operation with the maximum number of 
bunches (111) is preferable to reducing luminosity by reducing the number of
filled bunches because it allows sPHENIX to commission as it plans to run.  Initial 
stores should be zero crossing angle, both to begin operations with stores less likely to
be lost as well as to provide a direct comparison of vertex distribution between crossing angles of zero and the nominal 2 milliradians.

The superconducting solenoid should be operating at full field during the commissioning.
The minimum bias detector (MBD) gains are reduced by the magnetic field, and so tuning should take place at the field planned for physics operation.

\begin{itemize}

\item Initial stores should be one and a half weeks of 6 bunch stores with zero crossing
angle and up to 2~kHz of collisions.  
These collisions will be used for initial tune-up of timing and the MBD trigger.  
A simple ''blue logic'' trigger may be used initially for timing.
Several days may be required to begin operating the detector to provide time
for use of diagnostic instrumentation and to process data, but the stores can
be kept as long as practical.

\item The next stores should be two weeks of 111 bunch stores with zero crossing angle
and 1-5~kHz of collisions.  These stores will be used for tuning the MBD Level-1
trigger, which may require additional timing adjustments of the trigger primitives.
This phase will also require relatively short periods of data taking followed by analysis
and diagnostics.
Near the end of this period, the calorimeters can be turned on and timing them in can be
attempted if it has not been attempted during the previous commissioning period.
The second week should begin operation with the planned crossing angle.  
This should allow the first measurement of the vertex distribution with the
planned crossing angle, and begin any optimization of the ramp that may be necessary
while the tracking detectors, including the two inner silicon detectors,
are turned off.

\item We estimate a week of machine studies at this point for 
optimizing the crossing angle.  Careful detector-accelerator coordination will be important in this stage.

\item The next week of 111 bunch stores with non-zero crossing angle will be used for calorimeter timing and tuneup.
The crossing angle will allow us to assess the radiation dosage to the silicon photo-multipliers (SiPMs)
by measuring the leakage current with the monitoring system.

\item The next stores should be four weeks of 111 bunch stores with non-zero crossing angle
and 1-5 kHz of collisions.  
These stores will be used for initial operation of the tracking detectors, beginning with the TPC.  
The minimum bias trigger, developed in the previous weeks, will be used to trigger the
detector.  
It may be useful during this period to operate with zero magnetic field and/or very
low luminosity in order to collect data which can be used to align the tracking
detectors and characterize track distortions.
Some additional time may be necessary for data taking at zero field in order to
change the MBD high voltage for zero field.

\item The next two week period should be with 111 bunch stores, non-zero crossing
angle, and an increasing rate of minimum bias collisions, culminating in fills that
provide 15-20~kHz of collisions in order to stress test the data acquisition system
under running conditions.  

\end{itemize}

At the end of this nominal 11.5 week period of commissioning, the detector should be ready to take data from all detectors, triggered
according to plan for \auau, at the design rate of the data acquisition system.
Of course, data of publication quality requires more analysis than can be brought to bear 
on it while the collaboration is learning to operate and debug the apparatus, but it is
crucial to develop monitoring software which allows us to quickly assess the quality of
data as we take it.  We note that such commissioning time always has a significant uncertainty and run time 
flexibility in this first year of running will be required.

Additional trigger specific commissioning time is needed for each new collision species, particularly the physics-selective Level-1 triggers in the 2024 \pp and \pau running.   This commissioning time is built into the ramp up calculation for integrated luminosities in these periods.
We highlight that we have also included a 60\% uptime for the sPHENIX detector in 2023 and 2024, and then an 80\% uptime in 2025.