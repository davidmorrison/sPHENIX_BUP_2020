\chapter{sPHENIX Commissioning}
\label{chap:commissioning}

Commissioning sPHENIX with beams in RHIC should progress in stages with gradually
increasing luminosity.  Au+Au collisions are preferable for  commissioning due
to high multiplicities which can be used to assess the detector's health with 
larger occupancy.  Except for initial stores, operation with the maximum number of 
bunches (110 or 111) is preferable to reducing luminosity by reducing the number of
filled bunches because it allows sPHENIX to commission as it plans to run.  Initial 
stores should be zero crossing angle, both to begin operations with stores less likely to
be lost as well as to provide a direct comparison of vertex distribution between crossing angles
of 0 and 2 mr.

The superconducting solenoid should be operating at full field during the commissioning.
The MBD gains are reduced by the magnetic field, and so tuning should take place
at the field planned for physics operation.

\begin{itemize}

\item Initial stores should be 1.5 weeks of 6 bunch stores with zero crossing
angle and up to 2 kHz of collisions, although 110 bunch stores maybe satisfactory.  
These collisions will be used for initial tune-up of timing and the MBD trigger.  
A simple ''blue logic'' trigger may be used initially for timing.
Several days may be required to begin operating the detector to provide time
for use of diagnostic instrumentation and to process data, but the stores can
be kept as long as practical.

\item The next stores should be 2 weeks of 110 bunch stores with zero crossing angle
and 1-5 kHz of collisions.  These stores will be used for tuning the MBD LL1
trigger, which may require additional timing adjustments of the trigger primitives.
This phase will also require relatively short period of data taking followed by analysis
and diagnostics.
Near the end of this period, the calorimeters can be turned on and timing them in can be
attempted if it has not been attempted during the previous commissioning period.
The second week should begin operation with the planned crossing angle.  
This should allow the first measurement of the vertex distribution with the
planned crossing angle, and begin any optimization of the ramp that may be necessary
while the tracking detectors, including the two inner silicon detectors,
are turned off.

\item It may be necessary to take a week of machine studies at this point for 
optimizing the crossing angle.

\item The next week of 110 bunch stores with non-zero crossing angle will be used for calorimeter
timing and tuneup.
The crossing angle will allow us to assess the radiation dosage to the SiPM's
by measuring the leakage current with the monitoring system.

\item The next stores should also be 2-4 weeks of 110 bunch stores with non-zero crossing angle
and 1-5 kHz of collisions.  
These stores will be used for initial operation of the tracking detectors, beginning with the
TPC.  
The minimum bias trigger, developed in the previous weeks, will be used to trigger the
detector.  
It may be useful during this period to operate with zero magnetic field and/or very
low luminosity in order to collect data which can be used to align the tracking
detectors and characterize track distortions.
Some additional time may be necessary for data taking at zero field in order to
change the MBD  high voltage for zero field.

\item The next 2 week period should be with 110 bunch stores, non-zero crossing
angle, and an increasing rate of minimum bias collisions, culminating in fills that
provide 15-20 kHz of collisions in order to stress test the data acquisition system
under running conditions.  

\end{itemize}

At the end of this approximately twelve week period of commissioning, give or take for
good or ill fortune, the detector should be ready to take data from all detectors, triggered
according to plan for Au+Au, at the design rate of the data acquisition system.
Of course, data of publication quality requires more analysis than can be brought to bear 
on it while the collaboration is learning to operate and debug the apparatus, but it is
crucial to develop monitoring software which allows us to quickly assess the quality of
data as we take it.
