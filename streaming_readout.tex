
\section{Streaming Readout Upgrade for the sPHENIX trackers}
 
As proposed in Chapter~\ref{chap:beam_use_proposal}, the sPHENIX data taking will start with \auau collisions in 2023, followed \pp and \pau collisions in 2024.  

REMOVE TOO MANY SPECIFIC NUMBERS FROM HERE...

In the first three years of sPHENIX operation, the collaboration
envisions a run plan sampling 2 trillion \pp collisions (50~\pb) with
electron and jet triggers, as well as recording 140 billion M.B. \AuAu
collisions. Comparison of HF observables in heavy-ion collisions to
those in \pp reveals how the HF probes interact with QGP.  However,
due to the rareness of the HF signals in the \pp collisions, the
currently envisioned sPHENIX detector equipped with a traditional
triggered DAQ cannot efficiently sample HF production below a $p_T$ of
10~GeV$/c$.
 
A study~[SRO tech note] found that the sPHENIX tracking detectors can
be upgraded to record a large amount of \pp collisions via a streaming
readout upgrade in the data acquisition (DAQ) firmware and software.
(Embed something we want PAC to write in report) This chapter will
discuss the proposed implementation and benefit of this upgrade.

\subsection{Streaming Readout}
 
The currently envisioned sPHENIX experiment is designed to take large
statistics of calorimeter triggered events in the \pp collisions,
which will sample 2 trillion delivered \pp collisions in the vertex
acceptance of the silicon tracker (18\% of all
collisions)~\cite{something}. For observables that utilize the
calorimeters for analysis, a trigger can usually be designed, such as
leptonic decays at higher $p_T$, jets, photon, and their correlation
observables.  However, low-$p_T$ ($<10$~GeV$/c$) HF hadrons usually
decay hadronically and leave relatively small signals in the
calorimeters when compared with the underlying event. Therefore, they
cannot be efficiently collected via calorimeter triggers, which have a
hadron energy threshold of 10~GeV. Therefore, in the currently
envisioned sPHENIX detector, there is no efficient way to trigger on
such events in \pp collisions.  In the 15~kHz sPHENIX trigger
bandwidth, one would only reasonably request around few kHz of the
minimum bias \pp trigger bandwidth for this new program. Assuming a
vertex range selection purity of 50\%, a 1~kHz minimum bias trigger
leads to recording $2\times10^{-4}$ of the delivered luminosity. This
translates to rather limited statistics for these rare low-$p_T$ HF
signals as quantified in Table~\ref{tab:something}.
 
\subsection{Hybrid Trigger-Streaming Readout in Run24}
 
The upgraded DAQ is illustrated in Figure~\ref{fig:something} and
summarized in Table~\ref{tab:something}.  The FEE and DAQ hardware of
each of the three tracking detectors supports a streaming readout mode
and provides the capability needed for this upgrade. The main work
will focus on DAQ firmware and software development that enabling the
streaming capability.  This upgrade would enable the collection of
sufficient minimum bias in \pp events.  That is, 10\% delivered
luminosity (see next Chapter) or 200~billion events in the acceptance
of the vertex tracker, which is a factor of 500 improvement. This
dataset enables a comprehensive low-$p_T$ hadron program as discussed
here.  The analysis of these HF hadronic channels does not require
calorimeter information. Instead, as illustrated in
Figure~\ref{something}, they can be identified with the precision
tracking detectors of sPHENIX (at $p_T=5$~GeV$/c$, $\sigma(p)/p=1\%$,
$\sigma(DCA)<10$~$\mu$m ) via a combination of decay topology and
invariant mass, as demonstrated for even the busiest events of Au+Au
collisions through detailed simulation studies in~\cite{something}.

As summarized in Table~\ref{tab:something}, an upgraded DAQ for
sPHENIX trackers will enable the recording of 500 times greater
statistics of M.B. \pp collisions, which in turn will enable proper HF
measurements in this key kinematic region.

\subsection{Full streaming Readout in Run26}
 

\subsection{Data Preservation and Data Mining}

This upgrade will accumulate a large amount (200 billion) of minimum
bias polarized \pp collision without a trigger bias and with the full
sPHENIX tracking capability. After RHIC completes its scientific
mission at the end of the sPHENIX program, this would be a unique
dataset allowing future data mining for novel quantum effects such as
the quantum coherence in particle productions. These \pp data may be
critical in understanding the future $e+p/A$ collision data at an
EIC~\cite{something}.
