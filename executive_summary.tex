\chapter*{Executive Summary}
\label{executive_summary}
\setcounter{page}{1}

sPHENIX will be the first new collider detector at RHIC in over twenty years,
focusing measurement capabilities on heavy-ion collisions in this energy range
that have not been available before.  The experiment is a specific priority of
the DOE/NSF NSAC Long Range Plan, and the significance of its expected results,
complementing those coming from the LHC, was highlighted in the WG5 Heavy-Ion
input to the European Strategy for Particle Physics.  The effort to construct
the experiment is officially underway, with the DOE MIE project being granted
PD-2/3 approval in September 2019.  Additional detectors, funded as BNL capital
projects or realized as contributions from collaborating institutions, are also
under construction and will be integrated into the full experiment, augmenting
its science capabilities.  The sPHENIX resource-loaded schedule leads to a first
year of operation in 2023; the final year of sPHENIX operations in 2025 is
dictated by BNL's reference schedule for the EIC project.  This document
responds to a charge (see Appendix~\ref{chap:charge}) from the BNL NPP ALD to
detail the sPHENIX run plan during the years 2023--2025.

We describe a plan with three main features:
\begin{itemize}
\item A three year run plan that delivers the physics of the sPHENIX science proposal.
\item A first year that prioritizes proper commissioning of the
  sPHENIX detector.  Close coordination with C-AD will be required to
  validate beam handling strategies that will enable full exploitation
  of RHIC luminosity in years two and three of the plan.
\item A plan for additional running, should the occasion arise, that
  represents a unique opportunity for a massive, archival \auau and
  spin polarized \pp data set in the final years of RHIC operation.
\end{itemize}

The priority in the first year of sPHENIX operation will be to
properly commission and calibrate the detector.  Assuming this goes
well, we would then plan to take several weeks of \auau data.  The
priority oef the second year is to record the data for the \pp
baseline.  In a 24 cryo-week scenario, it would be difficult to go
beyond this plan.  In a sufficiently long run, we would switch species
and record \pAu data as well.  The third year of running focues on a
large statistics \auau run.  With the \pp baseline and the \auau data,
we would be able to carry out the goals of the sPHENIX science
proposal.  Table~\ref{exec:tab:summary} summarizes the data we would
expect to obtain. 

\renewcommand{\arraystretch}{1.9}
\addtolength{\tabcolsep}{-0.5pt}
\begin{table}[]
\centering
\caption{Summary of sPHENIX Beam Use Proposal for Years 2023-2025.
\label{tab:summary}}
\bigskip
\centering
\begin{tabular}{ | c | c | c | c | c | c | c | }
\hline
Year & Species & $\sqrt{s_{NN}}$ & Cyro  & Physics & Rec. Lum. & Samp. Lum. \\
     &         & [GeV]           & Weeks & Weeks   & $|z|<$10~cm & $|z|<$10~cm \\ \hline \hline

2023 & \auau   & 200 & 24 (28) & 9 (13) & 3.7 (5.7) \nb   & 4.5 (6.9) \nb  \\ \hline \hline 
2024 & $p^{\uparrow}p^{\uparrow}$     & 200 & 24 (28) & 12 (16) & 0.3 (0.4) \pb [5 kHz] & 73 (101) \pb  \\
     &                                &     &  & &  7.3 (10.1) \pb [10\%-$str$]&   \\ \hline
2024 & $p^{\uparrow}$+Au    & 200 & -- & 5 & 0.003 \pb [5 kHz]          & 0.11 \pb \\  
 &     &  &  &  &  0.01 \pb [10\%-$str$]         &   \\ \hline \hline
2025 & \auau   & 200 & 24 (28) & 20.5 (24.5) & 13 (15) \nb   & 21 (25) \nb  \\ \hline

\end{tabular}
\end{table}

\renewcommand{\arraystretch}{1.9}
\addtolength{\tabcolsep}{-0.5pt}
\begin{table}[h]
\centering
\caption{Summary of the sPHENIX Beam Use Proposal should a window of opportunity arise for the Years 2026-2027.\label{tab:summary2627}}
\bigskip
\centering
\begin{tabular}{ | c | c | c | c | c | c | c  | }
\hline
Year & Species & $\sqrt{s_{NN}}$ & Cyro  & Physics & Rec. Lum. & Samp. Lum. \\
     &         & [GeV]           & Weeks & Weeks   & $|z|<$10~cm & $|z|<$10~cm  \\ \hline \hline
     {\bf 2026} & $p^{\uparrow}p^{\uparrow}$   & 200 & 28 & 15.5      & 1.0 \pb [10 kHz]   & 130 \pb \\ 
      & & & & & 130~\pb [100\%-$str$] & \\ \hline
 --  & O+O    & 200 & -- & 2        & 13B +  42B [100\%-$str$] & 42B  \\ \hline
 --  & Ar+Ar   & 200 & -- & 2      & 13B + 29B [100\%-$str$] & 29B  \\ \hline \hline
{\bf{2027}} & \auau   & 200 & 28 & 24.5 & 30 \nb [100\%-$str$/DeMux]   & 31 \nb \\ \hline
\end{tabular}
\end{table}