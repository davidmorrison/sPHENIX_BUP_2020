\chapter*{Executive Summary}
\label{executive_summary}
\setcounter{page}{1}

sPHENIX will be the first new collider detector at RHIC in over twenty years,
focusing measurement capabilities on heavy-ion collisions in this energy range
that have not been available before.  The experiment is a specific priority of
the DOE/NSF NSAC Long Range Plan, and the significance of its expected results,
complementing those coming from the LHC, was highlighted in the WG5 Heavy-Ion
input to the European Strategy for Particle Physics.  The effort to construct
the experiment is officially underway, with the DOE MIE project being granted
PD-2/3 approval in September 2019.  Additional detectors, funded as BNL capital
projects or realized as contributions from collaborating institutions, are also
under construction and will be integrated into the full experiment, augmenting
its science capabilities.  The sPHENIX resource-loaded schedule leads to a first
year of operation in 2023; the final year of sPHENIX operations in 2025 is
dictated by BNL's reference schedule for the EIC project.  This document
responds to a charge (see Appendix~\ref{chap:charge}) from the BNL NPP ALD to
detail the sPHENIX run plan during the years 2023--2025.


