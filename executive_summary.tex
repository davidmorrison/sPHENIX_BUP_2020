\chapter*{Executive Summary}
\label{executive_summary}
\setcounter{page}{1}

sPHENIX will be the first new collider detector at RHIC in over twenty
years, focusing on measurement capabilities for heavy-ion collisions in
this energy range that have not been available before.  The experiment
is a specific priority of the DOE/NSF NSAC 2015 Nuclear Physics (NP) 
Long Range Plan, and the
significance of its expected results, complementing those coming from
the LHC, was highlighted in the WG5 Heavy-Ion input to the European
Strategy for Particle Physics.  sPHENIX will play a critical role in
the completion of the RHIC science mission by enabling qualitatively
new measurements of the microscopic nature of quark-gluon plasma.
These studies rely on very high statistics measurements of jet
production, jet substructure and open and hidden heavy flavor over an
unprecedented kinematic range at RHIC.  They are enabled by the high
rate capability and large acceptance of the detector, combined with
high precision tracking and electromagnetic and hadronic calorimetry.

The effort to construct the experiment is officially underway, with
the DOE Major Item of Equipment (MIE) project having been granted PD-2/3 approval in September
2019.  Additional detectors, funded as Brookhaven National Laboratory (BNL)
capital projects, such as
the micro-vertex tracker (MVTX), or realized as contributions from
collaborating institutions, are also under construction and will be
integrated into the full experiment, completing its science
capabilities.  The sPHENIX resource-loaded schedule leads to a first
year of operation in 2023; the final year of sPHENIX operations in
2025 is dictated by BNL's reference schedule for the Electron Ion Collider (EIC) project.
This document responds to a charge (see Appendix~\ref{chap:charge})
from the BNL NPP Associate Laboratory Director to detail the sPHENIX run plan during the years 2023--2025.

In the run plan described in this document, each of the three years
plays a critical role in fulfilling the science mission outlined in
the NP Long Range Plan:
\begin{itemize}
\item Year-1 (2023) serves to commission all detector subsystems and full
  detector operations, and to validate the calibration and
  reconstruction operations essential to delivering the sPHENIX
  science in a timely manner. Close coordination with C-AD will be
  required in the ramp-up of RHIC luminosity and optimization of beam
  operations to achieve these goals in a safe manner, enabling full
  exploitation of RHIC luminosity in Year-2 and Year-3. Year-1 will
  also allow collection of a \auau data set enabling sPHENIX to repeat
  and extend measurements of ``standard candles'' at RHIC.
\item Year-2 (2024) will see commissioning of the detector for \pp collisions
  (unless extended running beyond 28 cryo-weeks and successful \auau
  commissioning permits doing this already in Year-1) and collection
  of a large \pp reference data set, as well as a large \pAu data set
  for studies of cold QCD.
\item Year-3 (2025) is focused on collection of a very large statistics \auau
  data set for measurements of jets and heavy flavor observables with
  unprecedented statistical precision and accuracy.
\end{itemize}

Table~\ref{tab:exec:summary1} provides an overview of the data we
expect to obtain in Year-1 to Year-3 (2023 - 2025), as requested in the ALD charge.

\begin{table}[hbt!]
\centering
\caption{\label{tab:exec:summary1} Summary of sPHENIX Beam Use Proposal for the years
  2023--2025, as requested in the charge.  The values correspond to 24 cryo-week scenarios, while those in parentheses correspond to 28 cryo-week scenarios.    The 10\%-$str$ values correspond to a streaming readout of the tracking detectors.
  Full details are provided in
  Chapter~\ref{chap:beam_use_proposal}.} 
\bigskip \centering \begin{tabular}{ | c | c | c | c | c | c | c | }
\hline
Year & Species & $\sqrt{s_{NN}}$ & Cyro  & Physics & Rec. Lum. & Samp. Lum. \\
     &         & [GeV]           & Weeks & Weeks   & $|z|<$10~cm & $|z|<$10~cm \\ \hline \hline

2023 & \auau   & 200 & 24 (28) & 9 (13) & 3.7 (5.7) \nb   & 4.5 (6.9) \nb  \\ \hline \hline 
2024 & $p^{\uparrow}p^{\uparrow}$     & 200 & 24 (28) & 12 (16) & 0.3 (0.4) \pb [5 kHz] & 73 (101) \pb  \\
     &                                &     &  & &  7.3 (10.1) \pb [10\%-$str$]&   \\ \hline
2024 & $p^{\uparrow}$+Au    & 200 & -- & 5 & 0.003 \pb [5 kHz]          & 0.11 \pb \\  
 &     &  &  &  &  0.01 \pb [10\%-$str$]         &   \\ \hline \hline
2025 & \auau   & 200 & 24 (28) & 20.5 (24.5) & 13 (15) \nb   & 21 (25) \nb  \\ \hline

\end{tabular}

\end{table}

We also outline a plan for additional running in Year-4 to Year-5 (2026 - 2027), should the occasion
arise. This represents unique opportunities for collecting massive,
archival \apa and spin polarized \pp data sets in the final years of
RHIC operation, in addition to new geometry combinations.  Table~\ref{tab:exec:summary2} summarizes possibilities for additional run periods.

This document is organized as follows.    Chapter~\ref{chap:introduction} and Chapter~\ref{chap:project} provide brief summaries of the sPHENIX physics program and status of the sPHENIX project, respectively.    Chapter~\ref{chap:beam_use_proposal} details the Year-1 to Year-3 (2023-2025) Beam Use Proposal from sPHENIX including a detailed break down of cryo-weeks.    Chapters~\ref{chap:cad}, ~\ref{chap:commissioning}, ~\ref{chap:readout} detail the key inputs from the Collider-Accelerator Division, the commissioning plan for sPHENIX, and modest upgrades to the sPHENIX readout.    Chapter~\ref{chap:physics_projections} details the physics projections and deliverables from the Year-1 to Year-3 Beam Use Proposal.    We highlight that the full sPHENIX physics case is detailed in the original sPHENIX proposal, and here we focus on demonstrating that within this Beam Use Proposal those physics goals can be achieved.   Chapter~\ref{chap:beam_use_proposal_extra} details the potential opportunities from running in Year-4 to Year-5 (2026 - 2027) and finally Chapter~\ref{chap:summary} provides a brief summary. 

%We describe a plan with three main features:
%\begin{itemize}
%\item A three year run plan that delivers the physics of the sPHENIX science proposal.
%\item A first year that prioritizes proper commissioning of the
%  sPHENIX detector.  Close coordination with C-AD will be required to
%  validate beam handling strategies that will enable full exploitation
%  of RHIC luminosity in years two and three of the plan.
%\item A plan for additional running, should the occasion arise, that
%  represents a unique opportunity for a massive, archival \auau and
%  spin polarized \pp data set in the final years of RHIC operation.
%\end{itemize}
%
%The priority in the first year of sPHENIX operation will be to
%properly commission and calibrate the detector.  Assuming this goes
%well, we would then plan to take several weeks of \auau data.  The
%priority oef the second year is to record the data for the \pp
%baseline.  In a 24 cryo-week scenario, it would be difficult to go
%beyond this plan.  In a sufficiently long run, we would switch species
%and record \pAu data as well.  The third year of running focues on a
%large statistics \auau run.  With the \pp baseline and the \auau data,
%we would be able to carry out the goals of the sPHENIX science
%proposal.  Table~\ref{tab:exec:summary1} summarizes the data we would
%expect to obtain. 

\begin{table}[h]
\centering
\caption{ \label{tab:exec:summary2} Summary of the sPHENIX Beam Use Proposal should a window of
  opportunity arise for the years 2026--2027. 
  The values correspond to 28 cryo-week scenarios.
  The 100\%-$str$ values correspond to a streaming readout of the tracking detectors.
  Full details are provided in 
  Chapter~\ref{chap:beam_use_proposal_extra}.}  
\bigskip
\centering
\begin{tabular}{ | c | c | c | c | c | c | c  | }
\hline
Year & Species & $\sqrt{s_{NN}}$ & Cryo  & Physics & Rec. Lum. & Samp. Lum. \\
     &         & [GeV]           & Weeks & Weeks   & $|z|<$10~cm & $|z|<$10~cm  \\ \hline \hline
     {2026} & $p^{\uparrow}p^{\uparrow}$   & 200 & 28 & 15.5      & 1.0 \pb [10 kHz]   & 80 \pb \\ 
      & & & & & 80~\pb [100\%-$str$] & \\ \hline
       --  & O+O    & 200 & -- & 2        & 18~\nb & 37~\nb  \\ 
       & & & & & 37~\nb [100\%-$str$] & \\ \hline
 --  & Ar+Ar   & 200 & -- & 2      & 6~\nb  & 12~\nb  \\ 
        & & & & & 12~\nb [100\%-$str$] & \\ \hline \hline
{{2027}} & \auau   & 200 & 28 & 24.5 & 30 \nb [100\%-$str$/DeMux]   & 30 \nb \\ \hline
\end{tabular}

\end{table}

%
%Long Range Plan, sPHENIX will play a critical role in the completion of the RHIC science mission by enabling qualitatively new measurements of the microscopic nature of quark-gluon plasma. These studies rely on very high statistics measurements of jet production, jet substructure and open and hidden heavy flavor over an unprecedented kinematic range at RHIC.  They are enabled by the high rate capability and large acceptance of the detector, combined with high precision tracking and electromagnetic and hadronic calorimetry.
%
% September 2019.  Additional detectors, funded as BNL capital
%construction and will be integrated into the full experiment, completing its science capabilities.  The sPHENIX resource-loaded schedule leads to a first year of operation in 2023; the final year of sPHENIX operations in 2025 is dictated by BNL's reference schedule for the EIC project.  This document responds to a charge (see Appendix~\ref{chap:charge}) from the BNL NPP ALD to detail the sPHENIX run plan during the years 2023--2025.
%
%tlined in the NP Long Range Plan: 
%
%econstruction operations essential to delivering the sPHENIX science in a timely manner. Close coordination with C-AD will be required in the ramp-up of RHIC luminosity and optimization of beam operations to achieve these goals in a safe manner, enabling full exploitation of RHIC luminosity in Year-2 and Year-3. Year-1 will also allow collection of a \auau data set enabling sPHENIX to repeat and extend measurements of ``standard candles at RHIC. 
%ssful \auau commissioning permits doing this already in Year-1) and collection of a large p+p reference data set, as well as a large \pAu data set for studies of cold QCD. 
%rvables with unprecedented statistical precision and accuracy. 
%
%
%massive, archival \auau and spin polarized \pp data sets in the final years of RHIC operation.
%
%D charge, and table~\ref{tab:exec:summary2} summarizes possibilities for additional run periods.
