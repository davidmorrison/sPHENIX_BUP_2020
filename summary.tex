\chapter{Summary}
\label{chap:summary}

sPHENIX will be the first new collider detector at RHIC in over twenty
years,  performing very high statistics studies of jet
production, jet substructure and open and hidden heavy flavor over an
unprecedented kinematic range at RHIC.  The experiment
is a specific priority of the DOE/NSF NSAC 2015 Long Range Plan and 
will play a critical role in the completion of the RHIC science mission by enabling qualitatively
new measurements of the microscopic nature of quark-gluon plasma.
sPHENIX is distinguished by high rate capability and large acceptance, 
combined with high precision tracking and electromagnetic and hadronic calorimetry.

The effort to construct the experiment is officially underway, with
the DOE MIE project having been granted PD-2/3 approval in September
2019. The first year of data taking will be 2023; the final year of sPHENIX operations in
2025 as dictated by BNL's reference schedule for the EIC project.

Each run in this three year period plays a critical role in fulfilling the 
sPHENIX science mission outlined in the NP Long Range Plan:
\begin{itemize}
\item Year-1 serves to commission all detector subsystems and full
  detector operations, and to validate the calibration and
  reconstruction operations essential to delivering the sPHENIX
  science in a timely manner.  Year-1 will
  also allow collection of a \auau data set enabling sPHENIX to repeat
  and extend measurements of ``standard candles'' at RHIC.
\item Year-2 will see commissioning of the detector for \pp collisions
 and collection of a large \pp reference data set, as well as a large \pAu data set
  for studies of cold QCD.
\item Year-3 is focused on collecting a very large statistics \auau
 data set for measurements of jets and heavy flavor observables with
  unprecedented statistical precision and accuracy.
\end{itemize}

The collaboration also sees a strong physics case for additional running in Year-4 to Year-5 (2026 - 2027), should the occasion
arise. This represents unique opportunities for collecting massive,
archival \apa and spin polarized \pp data sets, and studying new geometries, in the final years of
RHIC operation. 
