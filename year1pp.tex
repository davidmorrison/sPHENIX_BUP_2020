\chapter{Beam Use Proposal (2023 \pp Commissioning Option)}
\label{chap:year1pp}

If more than 28 cryo-weeks of collider operation is available in Year-1 (2023) of operations, 
sPHENIX would consider requesting a short commissioning period in Year-1 (2023) with 
un-polarized 200~GeV \pp collisions.



Au+Au running is the highest priority for commissioning, but 4 additional 
weeks could be used for trigger development, a first look at the detector with
low multiplicity events, and potentially collect a sample of triggered photon
data which could be used to characterize the jet energy scale using photon-jet
events.
This run would be a tests of the detector and RHIC operation in advance of
the longer p+p run planned for Year-2 (2024). 
A minimum of four additional weeks of cryo weeks beyond the 24 weeks of Au+Au
commissioning would be used as follows:

\begin{itemize}

\item 1-2 weeks p+p setup
\item 1 week of ramp-up to design luminosity with non-zero crossing
angle
\item 1-2 weeks of trigger development
\item 1-2 weeks of data taking

\end{itemize}

Two weeks of data taking could potentially provide 10 \pb of triggered photon
p+p data with a non-zero crossing angle which would allow a first attempt 
at determining the jet energy scale with the sPHENIX detector.
We note that even 15 \pb of collected, triggered photon data would give a 1.5%
JES uncertainty in the golden photon-jet channel at 20 GeV as shown in Figure XXX.