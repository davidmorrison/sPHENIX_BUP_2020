\chapter{sPHENIX overview}
\label{chap:introduction}
\section{Science mission}
Over the last decades, experiments at RHIC and LHC have shown that collisions of heavy nuclei produce a novel hot and dense state of matter, called Quark-Gluon Plasma (QGP). These studies demonstrated that the QGP has properties that are unique among all forms of matter - in particular it is the most perfect liquid known. The QGP is a key example of a class of strongly coupled systems found recently in wide range of areas of physics, from string theory to condensed matter and ultra-cold atom systems.

While measurements have provided detailed knowledge of the QGP's macroscopic (long wavelength) properties, we do not yet understand how these properties arise from the fundamental interactions of its constituents, i.e., quarks and gluons governed by the laws of Quantum Chromodynamics (QCD).  In the 2015 Hot QCD Whitepaper and the US Nuclear Physics Long Range Plan (LRP)~\cite{Geesaman:2015fha}, one of two highest  priority goals in the field of Hot QCD was described as ``Probe the inner workings of QGP by resolving its properties at shorter and shorter length scales.
The complementarity of the two facilities [RHIC and LHC] is essential to this goal, as is a state-of-the-art jet detector at RHIC, called sPHENIX''~\cite{Geesaman:2015fha}.

\begin{figure}[htpb]
\begin{center}
\includegraphics[width=0.6\textwidth]{figs/sPhenix-Assembly-Labeled-With-Tracking-Shaded_with_Black_Lines-3600-BRIG.png}
\end{center}
\vspace{-0.5cm}
\caption{\label{fig:sPHENIX}
Engineering drawing (cutaway) of the sPHENIX detector. From the inside out the drawing shows tracking system, electromagnetic calorimeter and inner hadronic calorimeter, superconducting magnet and outer hadronic calorimeter.
A detailed discussion of the sPHENIX detector
subsytems can be found in~\cite{sPHENIX:TDR}}
\end{figure}

The sPHENIX physics program rests on a broad set of measurements using hard probes that are sensitive to the QGP microscopic structure over a range of length or momentum scales. These measurements include in particular studies of jet production and substructure, quarkonia suppression and open heavy flavor production and correlations.  

sPHENIX was proposed by the PHENIX collaboration in their 2010 decadal plan as an upgrade (or replacement) of the PHENIX experiment at RHIC. The physics case and detector case were further developed in the years leading up to the 2015 Nuclear Physics LRP. A detailed design proposal was completed in 2015~\cite{sPHENIX:2015irh}, and in early 2016 the current sPHENIX collaboration was formed. As of early 2020, sPHENIX has more than 320 members from 80 institutions in 13 countries. The project received DOE CD-0 approval in late 2016, CD-1/3A approval in 2018 and entered its
construction phase in fall 2019. The current schedule foresees commissioning of the detector in 2022 and start of physics data taking in early 2023.

\section{Key performance goals}

sPHENIX has been designed to allow high-statistics, high-resolution measurements for a broad
range of observables related to jet production and modification, quarkonia production
at high mass (or high \pt) and yields and correlations of heavy quark (charm \em and \em bottom) hadrons and heavy flavor tagged jets. Various benchmark plots for planned physics
measurements are shown below, based on the expected statistics in the run plan described
above, and the detector performance and acceptance seen in GEANT simulations.

An important result of the rate capability and resolution provided by the sPHENIX design
is a significantly increased kinematic range for single particle observables, relative
to prior measurements at RHIC.
Figure~\ref{fig:sPHENIX_MIE_master_AuAu_projections_tshirtcompare} compares
statistical projections for sPHENIX for various observables after the first sPHENIX data
taking campaign to the corresponding current $R_\mathrm{AA}$ measurements in central
Au$+$Au events by the PHENIX Collaboration.
While the existing measurements have greatly contributed to our
understanding of the QGP created at RHIC, the overall kinematic reach is
constrained to $< 20$~GeV even for the highest statistics
measurements. In contrast, the projected sPHENIX measurements reach sufficiently high \pt to
provide a large overlap with both low and and high \pt measurements at the LHC.

