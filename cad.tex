\chapter{C-AD Guidance}
\label{chap:cad}

For planning purposes in this document, we use luminosity projection numbers provided by the Collider-Accelerator Division (C-AD).   The version of the document titled 
``RHIC Collider Projections (FY 2017 –- FY 2027)'' utilized for this study is dated 12 August 2020 and utilizes knowledge gained from the Run-15 p+p and p+Au at 200~GeV running and the Run-16 \auau at 200 GeV running.   The document is available at:

\bigskip
{\color{blue}{http://www.rhichome.bnl.gov/RHIC/Runs/RhicProjections.pdf}} 
\bigskip

Note that the document linked above is periodically updated, so note the date tag.  In general, C-AD provides a minimum and maximum luminosity per week for each running period, as well as the fraction of collisions within a given z-vertex range. For calculating the integrated luminosity, we assume a ramp-up curve and then a steady-state physics running at the mean of the minimum and maximum in both luminosity and z-vertex fraction within $|z|<$10~cm (where a minimum and maximum are given).  

\subsection{Crossing Angle}

The original C-AD projections from three years ago for \auau 200~GeV collision rate as a function of time-in-store for the years 2023, 2025, and 2027 are shown in Figure~\ref{fig:auaulumcurves}.    The black curves are the collision rate for interactions at any longitudinal $z$ vertex position, while the red curves are the collision rate for interactions with $|z|<$10~cm.    The magenta curve corresponds to the design specified 15~kHz sPHENIX Level-1 trigger accept rate.
These projections are with zero crossing angle between the beams.   

The sPHENIX optimal acceptance for the inner tracking detectors is with collisions within $|z|<$10~cm.   Thus, it is clear that a majority of the collisions in this scenario have highly suboptimal tracking acceptance.   In principle these collisions far outside the optimal, i


\begin{figure}
\centering
\includegraphics[width=0.47\textwidth]{figs/figure_auauratestore_0mrad.pdf}
\includegraphics[width=0.47\linewidth]{figs/figure_auauratestore_2mrad.pdf}
\caption{(left) Estimated \auau at 200~GeV collision rate as a function of time in store for all collisions (black) and collisions within $\pm$ 10 cm (red).   The bottom to top set of curves in each color are for the C-AD projections in their document corresponding to 2023, 2025, 20278.
Also shown as a magenta band is the sPHENIX data acquisition rate of 15~kHz for reference.
These projections are with zero crossing angle between the beams. 
(right)
The same calculated quantities, now updated by C-AD, and for a 2 milliradian crossing angle between the beams.
\label{fig:auaulumcurves}}
\end{figure}



\subsection{Summary of C-AD Numbers}

Here are the basic inputs for the three collisions systems considered in this sPHENIX run plan (\pp, \pau, \auau all at 200 GeV).

\subsubsection{\auau at 200 GeV}

The C-AD projections are summarized in their document, Table 4.   We 
reproduce some of those key values here in Table~\ref{tab:auauspecs}.  

\begin{table}[h]
\centering
\caption{Summary of C-AD key values for \auau at 200 GeV running with the year labels as given in the C-AD document.
\label{tab:auauspecs}}
\bigskip
\begin{tabular}{ | c | c | c | c | c | c | c | c |}
\hline
Mode & \nb/wk & \nb/wk & $f_{z10}$ & $f_{z10}$ & ave/peak & peak rate & peak rate $\times f_{z10}$ \\ 
   	 & [min] & [max] & [min] & [max] &  & [max] & [max] \\ \hline
	Au+Au~(2022E) & 3 & 4.75 & 0.19 & 0.3 & 0.6 & 1.5E5 & 4.5E4 \\ \hline
	Au+Au~(2024E) & 3 & 7.02 & 0.3 & 0.3 & 0.6  & 2.2E5 & 6.6E4  \\ \hline
	Au+Au~(2026E) & 3 & 7.51 & 0.3 & 0.3 & 0.6  & 2.4E5 & 7.1E4  \\ \hline 
\end{tabular}
\end{table}

We consider running \auau in three calendar years (Year-1, Year-3, Year-5).   For the Year-1 run, we utilize the C-AD values they label as {\bf{2022E}}, where the minimum values correspond to those achieved in the 2016 run and the maximum are 58\% higher.   Wolfram Fischer has provided us with Figure~\ref{fig:auaustore} showing an example "best store" from Run-16 \auau at 200 GeV, where "best" is actually one of many stores that were reproduced with the same settings.  
The $f_{z10} = 0.19$ from Run-16 \auau is used as the minimum value, and 
$f_{z10} = 0.30$ is the projected maximum value.

\begin{figure}
\includegraphics[]{figure_cad_run16auau_examplestore}
\caption{Run-16 \auau at 200 GeV store.   The black line shows the luminosity (left y-axis units) as a function of time in store in hours.   The red dashed line shows the fraction of collisions within $\pm$ 10 cm. \label{fig:auaustore}}
\end{figure}

\subsubsection{\pp at 200 GeV}

The C-AD projections are summarized in their document, Table 6.   We 
reproduce some of those key values here in Table~\ref{tab:ppspecs}.  
Wolfram Fischer has provided us with Figure~\ref{fig:auaustore} showing an example "best store" from Run-15 \pp at 200 GeV, where "best" is actually one of many stores that were reproduced with the same settings.  

\begin{table}[h]
\centering
\caption{Summary of C-AD key values for \pp at 200 GeV running.
\label{tab:ppspecs}}
\bigskip
\begin{tabular}{ | c | c | c | c | c | c | c | c |}
\hline
Mode & \pb/wk & \pb/wk & $f_{z10}$ & $f_{z10}$ & ave/peak & peak rate & peak rate $\times f_{z10}$ \\ 
   	 & [min] & [max] & [min] & [max] &  & [max] & [max] \\ \hline
	p+p~(2023E) & 25 & 64 & 0.16 & 0.19 & 0.6 & 1.2E7 & 2.4E6 \\ \hline
	p+p~(2025E) & 25 & 64 & 0.19 & 0.19 & 0.6  & 1.2E7 & 2.4E6  \\ \hline
\end{tabular}
\end{table}

\begin{figure}
\includegraphics[]{figure_cad_run15pp200_examplestore}
\caption{Run-15 \pp at 200 GeV store.   The black line shows the luminosity (left y-axis units) as a function of time in store in hours.   The red dashed line shows the fraction of collisions within $\pm$ 10 cm.}
\end{figure}

\subsubsection{\pau at 200 GeV}

The C-AD projections are summarized in their document, Table 8.   We 
reproduce some of those key values here in Table~\ref{tab:pauspecs}.    

\begin{table}[h]
\centering
\caption{Summary of C-AD key values for \pau at 200 GeV running.
\label{tab:pauspecs}}
\bigskip
\begin{tabular}{ | c | c | c | c | c | c | c | c |}
\hline
Mode & \pb/wk & \pb/wk & $f_{z10}$ & $f_{z10}$ & ave/peak & peak rate & peak rate $\times f_{z10}$ \\ 
   	 & [min] & [max] & [min] & [max] &  & [max] & [max] \\ \hline
	p+Au~(2023E) & 0.14 & 0.35 & 0.17 & 0.25 & 0.6 & 2.8E6 & 6.9E5 \\ \hline
\end{tabular}
\end{table}










As placeholders, here are two figures JH received from C-AD.

\begin{figure}
    \centering
    \includegraphics[width=0.8\linewidth]{figs/figure_cad1_prelim.png}
    \caption{Caption}
    \label{fig:cad1}
\end{figure}

\begin{figure}
    \centering
    \includegraphics[width=0.8\linewidth]{figs/figure_cad2_prelim.png}
    \caption{Caption}
    \label{fig:cad2}
\end{figure}





\begin{figure}
    \centering
    \includegraphics[width=0.75\linewidth]{figs/figure_sphenix_auaustore.png}
    \caption{Caption}
    \label{fig:sphenixauaustore}
\end{figure}
